\documentclass[11pt]{article}
\usepackage{amsmath,amsthm,amssymb,fancyhdr,enumitem,tikz,tikz-cd,xcolor}
\usepackage[top=1in]{geometry}
\usepackage{pgfplots}
\usepackage{tikz}
\usepackage{tkz-fct}
\usetikzlibrary{arrows,matrix,cd,decorations.markings,positioning}
%\addtolength{\textwidth}{1in}

\theoremstyle{definition}
\newtheorem*{definition*}{Definition}
\newtheorem{problem}{}
\newcommand{\bp}{\begin{problem}}
\newcommand{\ep}{\end{problem}\bigskip}
\theoremstyle{theorem}
\newtheorem*{hint*}{Hint}
\newtheorem*{answer}{Answer}

\DeclareMathOperator{\id}{id}
\newcommand{\R}{\mathbb{R}}% the real numbers
\newcommand{\C}{\mathbb{C}}% the complex numbers
\newcommand{\Z}{\mathbb{Z}}% 


\begin{document}
\pagestyle{fancy}
\fancyfoot[R,C,L]{}
\fancyhead[R]{\large \textbf{Fall 2025}}
\fancyhead[L]{\large \textbf{Topology HW 2 - Due September 18}}

\newcommand{\Top}{\mathbf{Top}}
\newcommand{\Set}{\mathbf{Set}}




\begin{definition*}
A morphism $f:X\to Y$ in a category is a \emph{monomorphism} (mono) if $f\circ g=f\circ h$ implies $g=h$ for all $g,h:Z\to X$.  A morphism $f:X\to Y$ in a category is \emph{split} mono if there exists $g:Y\to X$ with $gf=\id_X$.  
\end{definition*}
\begin{definition*}
A morphism $f:X\to Y$ in a category is an \emph{epimorphism} (epi) if $g\circ f=h\circ f\Rightarrow g=h$ for all $g,h:Y\to Z$. A morphism $f:X\to Y$ in a category is \emph{split} epi if there exists $g:Y\to X$ with $fg=\id_Y$.  
\end{definition*}


\bp 
A subset $A$ of a topological space $X$ is \emph{dense} if and only if the smallest closed set containing $A$ is all of $X$.  
\begin{enumerate}[label=(\alph*)]
\item Prove that $A$ is dense iff for every nonempty open set $U$ in $X$, $U\cap A \neq \emptyset$.  
\item Prove that having a countable dense subset is a topological property (it's sometimes called \emph{second countable}.)
\end{enumerate}
\ep

\begin{problem}
On $\R^3$ define the \emph{Lumberjack metric} $d_L$ by
\[
d_L\big((x,y,z),(x',y',z')\big)=
\begin{cases}
|z-z'|, & \text{if } (x,y)=(x',y'),\\[2mm]
|z|+\,\sqrt{(x-x')^2+(y-y')^2}\,+|z'|, & \text{otherwise.}
\end{cases}
\]
\begin{enumerate}[label=(\alph*)]
\item Prove or disprove:  There is a countable dense subset of $\R^3$ with the lumberjack topology.
\item 
Prove that the $d_L$–topology is strictly finer than the Euclidean topology.
\item Prove or disprove:  The inclusion of $i:\R^2 \to \R^3$ defined by $i(x,y)=(x,y,0)$ defines an embedding of $\R^2$ with the usual topology into $\R^3$ with the lumberjack topology.


\end{enumerate}
\end{problem}


\begin{problem} Monomorphisms in $\Top$.
\begin{enumerate}[label=(\alph*)]
\item Prove that monomorphisms in $\Top$ are exactly the injective continuous maps.
\item Split monos in $\Top$ are given many names:  \emph{embeddings, sections, inclusions that have retractions}...  Just give a couple of example  of monomorpisms in $\Top$ that are not split.
\end{enumerate}
\end{problem}

\bp Define the \emph{Sierpiński space} $\mathbb S=\{0,1\}$ with topology $\{\varnothing,\{1\},\{0,1\}\}$.  
\begin{enumerate}[label=(\alph*)]
\item Prove that continuous maps $Y\to\mathbb S$ are exactly characteristic functions of open sets of $Y$.  That is, a function $\chi:Y\to\mathbb S$ is continuous iff $\chi^{-1}(\{1\})$ is open in $Y$.
\item Let $f:X\to Y$ be a set function between spaces $X$ and $Y$. Show that $f$ is continuous iff for every continuous $\chi:Y\to\mathbb S$, the composite $\chi\circ f:X\to\mathbb S$ is continuous. 
\end{enumerate}
\ep 

\pagebreak

\begin{problem} Epimorphisms in $\Top$.
\begin{enumerate}[label=(\alph*)]
\item Prove that if $f:X\to Y$ is an epi in $\Top$, then $f(X)$ is dense in $Y$. \emph{Hint:} Suppose $f(X)$ is not dense and construct distinct continuous maps $g,h:Y\to\mathbb S$ with $g\circ f=h\circ f$.
\item Give an explicit example showing that there exists a continuous $f:X\to Y$ with dense image that is \emph{not} epi in $\Top$. 
\end{enumerate}
\end{problem}

\begin{problem}More reasons the category $\Top$ is different than $\Set$.
\begin{enumerate}[label=(\alph*)]
\item In the category of sets, if there exists a monomorphism $f:X \to Y$ and a monomorphism $g:Y \to X$ then there exists an isomorphism $h:X \to Y$ (this is called the Canter-Schroder-Bernstein theorem).  Prove, by example, that there is no such theorem in topology.
\item In the category of sets, if $f:X \to Y$ is both a monomorphism and an epimorphism then $f$ is an isomorphism.  Prove, by example, that there are continuous functions $f:X\to Y$ that are both monic and epic but are not isomorphisms.
\end{enumerate}
\end{problem}


\bp Split epimorphisms are called retracts or retractions.  
\begin{enumerate}[label=(\alph*)]
\item Prove that a split epimorphisms in $\Top$ is a surjective quotient map.
\item Give an example to show that not all surjective quotient maps are split epis.
\end{enumerate}
\ep


\begin{problem} Let $X$ and $Y$ be topological spaces and let $A\subseteq X$ and $B\subseteq Y$ be subsets.  There are two ostensibly different ways to put a topology on the set $A\times B$.

\begin{enumerate}[label=(\alph*)]
\item First give $A$ and $B$ the \emph{subspace} topologies from $X$ and $Y$ and then put the product topology on $A\times B$. 
\item Second, give the subset $A\times B$ of $X\times Y$ the subspace topology inherited from the product topology on $X\times Y$.
\end{enumerate}
Prove or disprove:  these two constructions yield the same topology.
\end{problem}

\begin{problem}Let $q:X\to Y$ be a surjection and let $Z$ any space.  There are two ostensibly different ways to get a topology on $Y\times Z$.  
\begin{enumerate}[label=(\alph*)]
\item First take the quotient topology on $Y$ defined by $q$, then take the product with $Z$.
\item Give $X\times Z$ the product topology and then give $Y\times Z$ the quotient topology defined by $q\times \id_Z:X \times Z \to Y \times Z.$
\end{enumerate}

% \begin{enumerate}[label=(\alph*)]
% \item Describe the universal properties of these two constructions.
% \item \textbf{Counterexample} 
% Let $X=\mathbb{R}$, let $A=\{1/n:n\in\mathbb{N}\}$, and let $q$ collapse $A$ to a single point $*$, take $Z=\{0,1\}$ with the discrete topology...
% \end{enumerate}
Do you think these two constructions are the same?
\end{problem}

\end{document}