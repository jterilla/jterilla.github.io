\documentclass[11pt]{article}
\usepackage{amsmath,amsthm,amssymb,fancyhdr,enumitem,tikz,tikz-cd,xcolor}
\usepackage[top=1in]{geometry}
\usepackage{pgfplots}
\usepackage{tikz}
\usepackage{tkz-fct}
\usetikzlibrary{arrows,matrix,cd,decorations.markings,positioning}
%\addtolength{\textwidth}{1in}

\theoremstyle{definition}
\newtheorem*{definition*}{Definition}
\newtheorem{problem}{}
\newcommand{\bp}{\begin{problem}}
\newcommand{\ep}{\end{problem}\bigskip}
\theoremstyle{theorem}
\newtheorem*{hint*}{Hint}
\newtheorem*{answer}{Answer}

\DeclareMathOperator{\id}{id}
\newcommand{\R}{\mathbb{R}}% the real numbers
\newcommand{\C}{\mathbb{C}}% the complex numbers
\newcommand{\Z}{\mathbb{Z}}% 


\begin{document}
\pagestyle{fancy}
\fancyfoot[R,C,L]{}
\fancyhead[R]{\large \textbf{Fall 2025}}
\fancyhead[L]{\large \textbf{Topology HW 3}}

\newcommand{\Top}{\mathbf{Top}}
\newcommand{\Set}{\mathbf{Set}}
\newcommand{\N}{\mathbb{N}}

\paragraph{Instructions.}
These are some fun problems mixing product topology and dynamics, with a few optional challenge items. Work on them over the next week or so.  Aim to have this \textbf{checked off by September 25}.

\begin{itemize}
\item I will not collect full written solutions to everything. We’ll discuss a few in class; others may reappear on future sets or exams.
\item Discuss freely—working in a small group is encouraged. Please avoid AI tools; they tend to hand you the answers, which spoils the fun and the thrill of discovery.
\item By Sept 25, choose \emph{either}: (a) a quick oral check during office hours; (b) email me a one-page summary of which problems you solved and any questions you still have; or (c) if you want written feedback, submit solutions to problems 2b, 3, 7b, 8b, and 12a.
\end{itemize}

\section*{The Cantor Space}

Consider $\{0,1\}$ with the discrete topology.  The set $C=\{0,1\}^\N$ of binary sequences with the product topology is called the Cantor space.

\bp Prove the following facts about the Cantor space.
\begin{enumerate}[label=(\alph*)]
\item $C\cong C\times C$
\item $C\cong C \sqcup C$
\item Prove that $C$ has no isolated points.
\item $C$ is \emph{totally disconnected} meaning every connected component of $C$ is a singleton.
\item $C$ is metrizable.  (Hint: use $d(x,y)=2^{-\min\{n:\ x_n\ne y_n\}}$.)
\end{enumerate}
\ep


\bp Let $s:C \to C$ defined by $s(x_1, x_2, \ldots ) = x_2, x_3, \ldots$ be the backwards shift map.  Let $s^n$ denote the $n$-th iterate of $s$.  A point $x$ is \emph{periodic with period $n$} iff $s^n(x)=x$.  The \emph{least period} of a periodic point $x$ is the smallest positive integer $n$ for which $s^n(x)=x$.
\begin{enumerate}[label=(\alph*)]
\item Prove that $s$ is continuous.
\item Prove that the periodic points are dense in $C$.
\item Prove that the number of points with least period $n$ are equal to 
\[\sum_{d \vert n} \mu(d)2^{\frac{n}{d}}
\]
where $\mu$ is the M\"obius function from number theory.
\item Can you find a point $x\in C$ that has a dense orbit?
\end{enumerate}
\ep

\bp A homeomorphism $f:X \to X$ is \emph{topologically mixing} if and only if for all nonempty open sets $U, V$ there exists an integer $N\in \mathbb{Z}$ so that $f^n(U)\cap V \neq \emptyset$ for all $n\geq N$.  

If $f:X \to X$ is any continuous function (not necessarily a homeomorphism), you can say $f$ is \emph{topologically forward mixing} if and only if for all nonempty open sets $U, V$ there exists an integer $N\in \mathbb{N}$ so that $f^n(U)\cap V \neq \emptyset$ for all $n\geq N$.  

Prove that the shift map $s$ is topologically forward mixing.
\ep


\bp A homeomorphism $f:X \to X$ on a metric space is \emph{expansive} if and only if there exists a $\delta>0$ so that for all $x,y\in X$ there exists an $N\in \N$ so that $d(f^n(x),f^n(y))>\delta.$  Prove that the shift map $s$ is expansive.
\ep

\bp (Optional challenge) Look up the definition of topological entropy for a dynamical system and compute it for the shift map.
\ep


\section*{The $2$-adic integers}

Consider the following diagram of discrete topological spaces 
\[
\Z/2\Z \xleftarrow{\;\pi_1\;} \Z/4\Z 
\xleftarrow{\;\pi_2\;} \cdots 
\xleftarrow{\;\pi_{n-1}\;}\Z/2^{n}\Z
\xleftarrow{\;\pi_{n}\;} \Z/2^{n+1}\Z 
\xleftarrow{\;\pi_{n+1}\;} \cdots
\]
where $\pi_n$ is reduction mod $2^n$.  The $2$-adic integers are defined to be the \emph{limit} of this diagram.  That is the subspace of the product defined as follows:
\[\Z_2\ :=\ \lim \{\Z/2^{n+1}\xrightarrow{\;\pi_{n}\;} \Z/2^n \Z\}\ = \ \Big\{(x_n)_{n\ge1}\in\prod_{n\ge1}\Z/2^n\Z:\ x_n\equiv x_{n+1}\ (\mathrm{mod}\ 2^n)\Big\}.\]
So $(1,3,3,11,11,43,107, \ldots)$, for example,  could be the beginning of a typical sequence in $\Z_2$.

\bp 
\begin{enumerate}[label=(\alph*)]
\item Show $\Z_2$ is closed in $\prod \Z/2^n\Z$.
\item Show $\Z_2$ is totally disconnected.
\item Prove that $\Z_2$ has no isolated points.
\item Prove that $\Z_2$ is metrizable. 
\end{enumerate}
\ep

\bp Consider the map $T:\Z_2\to\Z_2$, $T(x)=x+1$.
\begin{enumerate}[label=(\alph*)]
\item Define a metric on $\Z_2$ by $d(x,y)=2^{-\min\{n:\ x_n\ne y_n\}}$.  Prove that $T$ is a homeomorphism and an isometry.
\item Prove that the orbit of every point is dense in $\Z_2$ and that $T$ has no periodic points.
\end{enumerate}
\ep


\bp Define a map $\Phi:C\to \Z_2$ from the Cantor Set to the $2$-adic integers as follows: \[\Phi_{\mathrm{seq}}\big((x_0,x_1,\ldots)\big) \;=\; (r_k)_{k\ge1},
\]
where
\[
r_k \;=\; \left(\,\sum_{n=0}^{k-1} x_n\,2^n\,\right) \mod {2^k} \;\in\; \Z/2^k\Z.
\]
So, for example $\Phi(1,1,0,1,0,0,\ldots) = (1, 3, 3, 11, \ldots )$.  To see this, look at 
\begin{align*}
1\times2^0 \mod 2 &= 1\\
1\times2^0+1\times2^1\mod 4&=3\\
1\times2^0+1\times2^1+0\times2^2\mod 8 &=3\\
1\times2^0+1\times2^1+0\times2^2+1\times 2^3\mod 16 &= 11
\end{align*}
\begin{enumerate}[label=(\alph*)]
\item Prove $\Phi$ is a homeomorphism.
\item Define the \emph{odometer} $\tau:C\to C$ to be $\tau=\Phi^{-1}T\circ \Phi$, i.e. the transport of the map $T$ via the isomorphism $\Phi$: 
\[
\begin{tikzcd}
C \arrow[r, "\tau"] \arrow[d, "\Phi"] & C \arrow[d, "\Phi"] \\
\Z_2 \arrow[r, "T"] & \Z_2
\end{tikzcd}
\]
\end{enumerate}
Explain how $\tau$ works explicitly as a map from $C\to C$.
\ep



\bp Prove that the odometer $\tau$ is not topologically mixing, and is not expansive.
\ep


\bp (Optional challenge) Look up the definition of topological entropy for a dynamical system and compute it for the odometer.
\ep


\section*{The torus}

\bp The group $\Z^2$ acts on $\R^2$ by $(n,m)\cdot (x,y)\mapsto (x+n,y+m)$ defining an equivalence relation $(x,y)\sim(x+n,y+m)$ for $(n,m)\in \Z^2$.  Define the torus $T^2$ to be the quotient \[T^2:=\R^2/\Z^2.\]   Let $p:\R^2 \to T^2$ be the quotient map $(x,y)\mapsto [(x,y)].$  
\begin{enumerate}[label=(\alph*)]
\item Show that the map $\psi:\R^2 \to S^1\times S^1$ defined by $\psi:(x,y)\mapsto (e^{2\pi i x},e^{2\pi i y})$ induces a homeomorphism $\R^2/\Z^2 \xrightarrow{\;\cong\;} S^1\times S^1$.
\item For any $2$ by $2$ matrix with integer entries $A\in M_2(\Z)$, define $f_A:T^2\to T^2$ by $f_A([v])=[Av]$. 
\[
\begin{tikzcd}
\R^2 \arrow[r, "\scriptstyle A"] \arrow[d, "\scriptstyle p"] & \R^2 \arrow[d, "\scriptstyle p"] \\
T^2 \arrow[r, "\scriptstyle f_A"] & T^2
\end{tikzcd}
\]
 Check the details to understand why $f_A$ is well defined and continuous.
\end{enumerate}
\ep

\bp Let $A=\begin{pmatrix}2&1\\[2pt]1&1\end{pmatrix}$ and $f=f_A$.
\begin{enumerate}[label=(\alph*)]
\item Show that $f:T^2\to T^2$ is a homeomorphism.
\item Find the periods of $\left[\left(0,0\right)\right]$,$\left[\left(0,\frac{1}{4}\right)\right]$, and $\left[\left(\frac{1}{5},\frac{2}{5}\right)\right]$.
\item Compute the eigenvalues $\lambda>1$ and $\lambda^{-1}<1$ and corresponding eigenvectors $v^u,v^s\subset\R^2$.  Define lines $E^u=\R v^u$ and $E^s=\R v^s$.
\item For any point $[x]\in T^2$, choose a lift $\tilde{x}\in \R^2$ and and define the \emph{unstable/stable lines through $[x]$} by
\[
W^u([x])=p(\tilde x+E^u),\qquad W^s([x])=p(\tilde x+E^s).
\]
Show these are independent of the choice of lift and $f$-invariant:
$f(W^u([x]))=W^u(f([x]))$ and $f^{-1}(W^s([x]))=W^s(f^{-1}([x]))$.
\item Prove that $f_A$ is expansive.
\end{enumerate}
\ep

\bp (Optional challenge)  Again, $A=\begin{pmatrix}2&1\\[2pt]1&1\end{pmatrix}$ and $f=f_A$.
\begin{enumerate}[label=(\alph*)]
\item Show that periodic points of $f$ are dense in $T^2$.
\item Show that $f$ is topologically mixing.
\end{enumerate}
\ep

\end{document}